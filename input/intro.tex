Entscheidungsbäume, wie sie aus der Medizin oder Biologie bekannt sind
können zur Erklärung von Zusammenhangsstrukturen in Daten verwendet
werden. Das funktioniert sowohl für kategoriale Zielvariablen
(Klassifizierungsbäume) als auch für numerische Zielgrößen
(Regressionsbäume). Bäume teilen den Merkmalsraum X, der die Zielgröße
y erklären soll in Partitionen auf. Die Partitionen werden so gewählt,
dass die Teilmengen innerhalb der Partitionen möglischt homogen
bezüglich der Zielgöße y sind und möglichst heterogen zu anderen
Partitionen. Bei dem CART-Ansatz werden die Merkmale binär
gesplittet. Außerdem ist der Algorithmus rekursiv definiert, so dass
bereits für einen Split benutze Variablen auch bei einem späteren
Split wiederverwendet werden könnten. [REF]
CART arbeitet wie folgt: 
Der Merkmalsraum wird rekursiv binär partitioniert und anschließend
wird in den neu entstandenen Partitionen ein konstantes Modell
angepasst. Welche Variable aus dem Merkmalsraum für den Split
verwendet wird, wird durch ein Kriterium entschieden. Dafür wird ein
Informationskriterium verwendet. 

